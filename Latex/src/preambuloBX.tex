% !TEX encoding = IsoLatin9

\documentclass[aspectratio=54, 9pt]{beamer}
\usetheme[progressbar=frametitle]{metropolis}
\usepackage{graphicx}
\usepackage{xcolor}

% Define los colores del Banco de M�xico
\definecolor{bmxblue}{RGB}{0,56,168}     % Azul del Banco de M�xico
\definecolor{bmxgray}{RGB}{131,139,141}  % Gris del Banco de M�xico
\definecolor{bmxgreen}{RGB}{0,153,68}    % Verde del Banco de M�xico
\definecolor{bmxorange}{RGB}{255,102,0}  % Naranja del Banco de M�xico

% Aplica los colores al tema Metropolis
\setbeamercolor{alerted text}{fg=bmxorange}
\setbeamercolor{frametitle}{bg=bmxblue, fg=white}
\setbeamercolor{title separator}{fg=bmxblue}
\setbeamercolor{progress bar}{fg=bmxgreen, bg=bmxgray}
\setbeamercolor{block title}{bg=bmxblue, fg=white}
\setbeamercolor{block body}{bg=bmxgray!20, fg=black}
\setbeamercolor{background canvas}{bg=white}

\usepackage{appendixnumberbeamer}
\usepackage{booktabs}
\usepackage[sfdefault]{FiraSans} %% option 'sfdefault' activates Fira Sans as the default text font
\usepackage[T1]{fontenc}
\renewcommand*\oldstylenums[1]{{\firaoldstyle #1}}
\newcommand\norm[1]{\left\lVert#1\right\rVert}           %% simbolo de la norma
\usepackage{mwe}     % For dummy images
\usepackage{lmodern} % To suppress some warnings
%-----------------------------------------------------------------
%-----------------------------------------------------------------
% Algunos paquetes.
\usepackage[spanish, mexico]{babel}
\usepackage[latin1]{inputenc}
\usepackage{amsthm}
\usepackage{enumerate}
\usepackage{amsmath}
\usepackage{hyperref}
\usepackage{subfig}                                            
\usepackage{tikz}
\usepackage{ragged2e}
\setbeamertemplate{bibliography item}[text] % a�adir referencias sin citarlas
\usepackage{appendixnumberbeamer}
\justifying
%\setbeamercolor{math text}{fg=blue}
\usepackage{pgf,tikz}
\usepackage{multirow}                      %%Para unir filas en tablas
\usetikzlibrary{arrows}
\newtheorem{defi}{Definici�n}
\newtheorem{teo}{Teorema}
\newtheorem{prop}{Proposici�n}
\RequirePackage{xcolor}
%\usecolortheme[RGB={99,99,99}]{structure} 
\definecolor{OwlRed}{RGB}{222,45,38}
\definecolor{OwlGreen}{RGB}{90, 168, 0}
\definecolor{OwlBlue}{RGB}{49,130,189}
\definecolor{OwlYellow}{RGB}{ 242, 147,  24}
\newcommand{\azul}[1]{\textcolor{OwlBlue}{#1}}
\newcommand{\amarillo}[1]{\textcolor{OwlYellow}{#1}}
\newcommand{\rojo}[1]{\textcolor{OwlRed}{#1}}
\newcommand{\verde}[1]{\textcolor{OwlGreen}{#1}}
\newcommand{\bs}[1]{\boldsymbol{#1}}

\usepackage{multicol}
\usepackage{multirow}
%\usepackage[table,xcdraw]{xcolor}


\usepackage{listings}
\definecolor{codegreen}{rgb}{0,0.6,0}
\definecolor{codegray}{rgb}{0.5,0.5,0.5}
\definecolor{codepurple}{rgb}{0.58,0,0.82}
\definecolor{backcolour}{rgb}{0.95,0.95,0.92}

\lstdefinestyle{mystyle}{
    backgroundcolor=\color{backcolour},   
    commentstyle=\color{codegreen},
    keywordstyle=\color{magenta},
    numberstyle=\tiny\color{codegray},
    stringstyle=\color{codepurple},
    basicstyle=\ttfamily\footnotesize,
    breakatwhitespace=false,         
    breaklines=true,                 
    captionpos=b,                    
    keepspaces=true,                 
    numbers=left,                    
    numbersep=5pt,                  
    showspaces=false,                
    showstringspaces=false,
    showtabs=false,                  
    tabsize=2
}

\lstset{style=mystyle}

